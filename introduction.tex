\section{Introduction}\label{section:introduction}

\kant[1] 

In this paragraph, we show how you can reference an equation, such as Gauss's Law in integral form shown in Equation~\ref{eq:gauss_law_integral_form}. 

% =======================================================
\begin{equation}
\label{eq:gauss_law_integral_form}
\mathbf{E} \cdot \mathrm{d}\mathbf{S} ={\frac{1}{\varepsilon_{0}}}\iiint_{\Omega}\rho\,\mathrm{d}V
\end{equation}
% =======================================================

\kant[2] 

In this paragraph, I am including some numerical results with and without units, inline with the text. Some examples would be; \complexnum{1+-2i} for complex numbers,
\num{.3e45}~\unit{\kilogram\metre\per\second} for a scientific notation, and 
\qtylist{0.13;0.67;0.80}{\milli\metre} for a list of answers. Below we show an answer in a math equation environment.

% =======================================================
\begin{equation}
\label{eq:numbers_and_units}
\mathbf{E} = \num{1.2e-12}\ \unit{kg.m.s^{-1}}
\end{equation}
% =======================================================


The Maxwell/Faraday equation in integral form is shown in equation~\ref{eq:maxwell_faraday_equation_integral_form}

% =======================================================
\begin{equation}
\label{eq:maxwell_faraday_equation_integral_form}
\oint_{\partial\Sigma} \mathbf{E} \cdot \mathrm{d}{\mathbf{\ell}}=-{\frac {\mathrm{d}}{\mathrm{d}t}} \iint_{\Sigma}\mathbf{B}\cdot\mathrm{d}\mathbf{S} 
\end{equation}
% =======================================================

Using \texttt{biblatex} you can display bibliography divided into sections,  depending of citation type. Let's cite! The Einstein's journal paper \cite{einstein} and the Dirac's book \cite{dirac} are physics related items. Next, \textit{The \textsc{latex} Companion} book \cite{latexcompanion}, the Donald Knuth's website \cite{knuthwebsite}, \textit{The Comprehensive Tex Archive Network} (CTAN) \cite{ctan} are \textsc{latex} related items; but the others Donald Knuth's items \cite{knuth-fa,knuth-acp} are dedicated to programming. 


