\section{Introduction}\label{section:introduction}

\kant[1] 

We can deduce that, then, the noumena are just as necessary as, so regarded, the practical employment of the objects in space and time, such as in Equation~\ref{eq:gauss_law_integral_form}. It is obvious that the manifold has nothing to do with our ideas; with the sole exception of the employment of th noumena, natural reason, in natural theology, is the mere result of the power of time, a blind but indispensable function of the soul.
. 

% =======================================================
\begin{equation}
\label{eq:gauss_law_integral_form}
\mathbf{E} \cdot \mathrm{d}\mathbf{S} ={\frac{1}{\varepsilon_{0}}}\iiint_{\Omega}\rho\,\mathrm{d}V
\end{equation}
% =======================================================

\kant[1] 

The paralogisms of pure reason are just as necessary as, in all theoretical sciences, our knowledge. The things in themselves constitute a body of demonstrated doctrine, and some of this body must be known a posteriori; \complexnum{1+-2i} for complex numbers, \num{.3e45}~\unit{\kilogram\metre\per\second} for a scientific notation, and \qtylist{0.13;0.67;0.80}{\milli\metre} for a list of answers. Therefore, the things in themselves can not take account of the transcendental aesthetic, as shown in Equation~\ref{eq:numbers_and_units}

% =======================================================
\begin{equation}
\label{eq:numbers_and_units}
\mathbf{E} = \num{1.2e-12}\ \unit{kg.m.s^{-1}}
\end{equation}
% =======================================================

Where, transcendental Deduction exists in the Ideal. To avoid all misapprehension, it is necessary to explain that pure reason (and it is obvious that this is true) is the key to understanding the transcendental unity of apperception.~\ref{eq:maxwell_faraday_equation_integral_form}

% =======================================================
\begin{equation}
\label{eq:maxwell_faraday_equation_integral_form}
\oint_{\partial\Sigma} \mathbf{E} \cdot \mathrm{d}{\mathbf{\ell}}=-{\frac {\mathrm{d}}{\mathrm{d}t}} \iint_{\Sigma}\mathbf{B}\cdot\mathrm{d}\mathbf{S} 
\end{equation}
% =======================================================

\kant[1] 

Using \texttt{biblatex} you can display bibliography divided into sections,  depending of citation type. Let's cite! The Einstein's journal paper \cite{einstein} and the Dirac's book \cite{dirac} are physics related items. Next, \textit{The \textsc{latex} Companion} book \cite{latexcompanion}, the Donald Knuth's website \cite{knuthwebsite}, \textit{The Comprehensive Tex Archive Network} (CTAN) \cite{ctan} are \textsc{latex} related items; but the others Donald Knuth's items \cite{knuth-fa,knuth-acp} are dedicated to programming. 


